% \newglossaryentry{mot}
% {
%     name=mot,
% 	plural=mots,
% 	description={concaténation de lettres de l'alphabet}
% }
\newglossaryentry{aws} 
{
    name={Amazon Web Services (\textsc{AWS})},
    description={
Division d'Amazon spécialisée dans les services de cloud computing à la demande pour les entreprises et particuliers. AWS met à disposition de ses clients des machines, disponible à tout moment, via Internet
    },
    first={Amazon Web Services (\textsc{AWS})},
    text={\textsc{AWS}}
}


\newglossaryentry{poc} 
{
    name={Proof of Concept (\textsc{POC})},
    description={« Démonstration de faisabilité » en français. Réalisation courte et incomplète d'un projet pour en démontrer la faisabilité. La traduction française est rarement utilisée dans le langage professionnel courant},
    first={Proof of Concept (\textsc{POC})},
    text={\textsc{POC}}
}

\newglossaryentry{dsi} 
{
    name={Directeur des systèmes d'information (\textsc{DSI})},
    description={
Directeur des systèmes d'information (DSI), ou directeur informatique (DI) est responsable de l'ensemble des composants matériels et logiciels du système d'information, ainsi que du choix et de l'exploitation des services de télécommunications mis en œuvre
},
    text={\textsc{DSI}}
}

\newglossaryentry{front-end}
{
    name={Front-end},
	description={
Le développeur \textit{front-end} -- ou web frontal -- est en charge des parties des applications ou des sites web qu'un utilisateur peut voir et avec lesquelles il peut intéragir directement 
	},
	text={\textit{front-end}}
}

\newglossaryentry{back-end}
{
    name={Back-end},
	description={
\og Arrière-plan \fg en informatique. Correspond à la couche d'accès aux données. En opposition à la couche \textit{front-end} de \og présentation \fg des données
	},
	text={\textit{back-end}}
}

\newglossaryentry{full-stack}
{
    name={Full-stack},
	description={
Qui peut réaliser toutes les tâches d’un développeur, à n’importe quel niveau de la « pile technique ». Couramment utilisé pour parler d'un développeur à la fois \textit{front-end} et \textit{back-end}
	},
	text={\textit{full-stack}}
}

\newglossaryentry{ux}
{
    name={UX},
	description={
L'expérience utilisateur (\textit{user experience}) est la qualité du vécu de l'utilisateur dans des environnements numériques ou physiques. Synonyme de ce que l'on qualifiait autrefois d'ergonomie et d'utilisabilité
	},
	text={\textsc{UX}}
}


\newglossaryentry{mvp} 
{
    name={Minimum Viable Product (\textsc{MVP})},
    description={
    Un produit minimum viable est une version d'un produit comprenant seulement des fonctionnalités de base permettant aux premiers clients de donner leurs retours afin d'en améliorer le futur développement
    },
    first={\textit{Minimum Viable Product} (\textsc{MVP})},
    text={\textsc{MVP}}
}

\newglossaryentry{inter-staffing} 
{
    name={Inter-staffing},
    description={
    Pour un consultant, période entre deux missions de conseil pour un client
    },
    text={\textit{inter-staffing}}
}

\newglossaryentry{pas} 
{
    name={Plan d’Assurance Sécurité (PAS)},
    description={Document précisant les exigences de sécurité définies par le maître d’ouvrage en ce qui concerne leur organisation et leur système d’information},
    first={Plan d’Assurance Sécurité (\textsc{PAS})},
    text={\textsc{PAS}}
}

\newglossaryentry{devops} 
{
    name={DevOps},
    description={Combinaison de pratiques et d'outils qui améliore la capacité d'une entreprise à livrer des applications et des services à un rythme élevé.
    Avec ce modèle, les équipes de développement et d'opérations ne sont plus isolées},
    text={\textit{DevOps}}
}

\newglossaryentry{scrum} 
{
    name={Scrum},
    description={TODO},
    text={\textit{Scrum}}
}

\newglossaryentry{agile} 
{
    name={Manifeste Agile},
    description={TODO},
    text={manifeste agile}
}