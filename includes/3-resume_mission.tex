\entreprise m'a recruté en tant qu'apprenti informaticien en juin 2019.
Ma mission générale est celle de ce que l'on appelerait un développeur web 
\gls{front-end}\footnote{\glsdesc{front-end}}
et 
\gls{back-end}\footnote{\glsdesc{back-end}} 
-- parfois appelé 
\gls{full-stack}\footnote{\glsdesc{full-stack}}
par abus de langage.

Ainsi, lors de la conception d'applications, je m'occupe aussi bien de la création de la couche de présentation des données -- l'interface web --, que de l'accès, du traitement et du stockage de celles-ci.

De plus, j'ai eu l'opportunité de participer à la création des outils \textit{DevOps} et de l'infrastructure \textit{cloud} via les
\gls{aws}\footnote{\glsdesc{aws}}.

En effet, une fois les applications créées par les développeurs, il faut les rendre accessibles aux utilisateurs. C'est pourquoi des \textit{pipelines} de déploiement ont été mises en place avec l'équipe. Elles facilitent et automatisent le déploiement des applications pour les développeurs.

Ces applications doivent évidemment être hébergées quelque-part pour être accessibles. \entreprise utilise \gls{aws} comme infrastructure \textit{cloud}. Je me suis donc également formé à leurs outils et services. D'ailleurs, au fil de mes missions chez \entreprise on m'a donné l'opportunité de travailler encore plus sur l'infrastructure \textit{cloud} \gls{aws}.

\newpage{}

Ce mémoire se concentrera sur la mission \emph{Bouclage de production} réalisée pour le compte de la \sncf de juillet 2019 à février 2020.

La \sncf sépare le réseau \tgv français en plusieurs axes, comme l'axe Atlantique, Nord ou encore Sud-Est.
Pour connaître l'état d'une ligne de \tgv, les salariés utilisent les données provenant de plusieurs applications internes, comme Octopus, Concerto, Résarail, etc.

Le \og bouclage de production \fg permet aux directions des axes de rassembler les données des différentes applications afin de reconstituer l'ensemble de la chaîne de production ferroviaire.
Après l'ouverture des ventes, cela permet de s'assurer de la disponibilité des ressources, et d'identifier puis suivre les problèmes de l'une des briques de la chaîne de production.

En juillet 2019, la \sncf ne mettait à disposition aucun outil aux directions des axes pour effectuer ce bouclage de production. Ainsi, chacun avait sa propre manière de faire.

Une solution développée en interne par l'axe Sud-Est permet de répondre à ce besoin.
Cette solution s'appuie sur des fichiers Excel et le développement de macro.
Cependant, elle nécessite de nombreuses actions manuelles pour générer le tableur de synthèse, et elle ne répond pas aux enjeux de robustesse et de maintenabilité attendus sur ce type d'outil.
Enfin, elle répond aux besoins de l'axe Sud-Est mais ne prend pas en compte les besoins spécifiques des autres axes.

La \sncf voudrait donc transformer la solution de l'axe Sud-Est pour l'automatiser, la rendre pérenne, et pouvoir l'utiliser sur tous les axes.

% projet en 3 lots.

% Juillet 2019
% Novembre 2019
% Décembre 2019

% PARLER DES LOTS DANS PLANNING. PAS ICI.

% parler du mvp. du poc. de pourquoi. le but du projet.

% AXES: 
% TRAFIC TGV : 4 TRAINS SUR 5 EN MOYENNE
% Axe Est : 7 trains sur 10
% Axe Atlantique : 9 trains sur 10
% Axe Nord : trafic normal
% Axe Sud-Est : 4 trains sur 5
% Intersecteurs : 3 trains sur 4
% Ouigo : 4 trains sur 5
