J'ai effectué ma première année de cycle ingénieur en formation traditionnelle.\\
L'\epita propose pléthore de projets informatiques, dont un consacré au web en fin d'année.

Ce dernier projet se veut également très proche de la réalité du travail et demandait aux élèves de suivre le protocole théorique de la gestion de projet : appel d'offres, planning, réunions fréquentes avec le \og client \fg{} -- joué par les assistants éducatifs.

J'ai par la suite pu intégrer la formation en apprentissage du cycle ingénieur de l'\epita. J'ai travaillé 1 an chez Milleis Banque -- anciennement Barclays -- sur des projets web, déjà sous la tutelle de \damien.

Ainsi, tous les projets sur lesquels j'ai travaillé jusqu'à présent étaient des projets internes à mon entreprise -- ou à mon école.
C'était la première mission que j'effectuais pour un client.

C'est donc un nouveau challenge pour moi, puisqu'un client sera moins souple qu'un projet commandé en interne.
De plus, comme la \df est assez jeune, beaucoup d'outils seront à mettre en place en parallèle de la conception du projet lui-même. Notamment la gestion du système d'information et l'hébergement des applications créées.

Cette mission s'appuie ainsi sur les compétences déjà acquises à l'\epita et lors de mes dernières missions chez Milleis Banque. Mais viennent s'ajouter à celles-ci de nouvelles qui compléteront mes compétences d'ingénieur.