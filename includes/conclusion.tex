
% \begin{quote}
% Repositionnez la problématique de l’entreprise ou du sujet de la mission par rapport à un état de
% l’art du marché. Il s’agira par exemple dans le cas d’un produit, d’une méthode d’un standard, de le
% positionner par rapport à d’autres alternatives en démontrant votre maîtrise du domaine.
% Le but de ce chapitre est de permettre au jury d’identifier et de quantifier le reste à faire d’ici la fin de
% votre mission et les résultats à venir et de mesurer la valeur ajoutée que vous avez apportée à
% l’entreprise et de comprendre en quoi vous avez répondu au besoin de l’entreprise.
% \end{quote}

\subsection{Intérêt de la mission pour l'entreprise}

% \begin{quote}
% Vous détaillerez l’intérêt à court et long terme de la mission pour l’entreprise, les
% perspectives ouvertes par ce sujet et la valeur ajoutée qu’elle peut en retirer pour le produit,
% le service et l’entreprise - dans le cas d’une SSII, vous mettrez en évidence ces mêmes
% perspectives tant pour la société de service que pour le client final.
% \end{quote}

La \df a été créée en mai 2019 et doit encore prouver son utilité aux yeux de la direction de \tnp qui a fait confiance à \damien et \gil pour lancer cette sous-offre de Digital \& Solutions.

\emph{Bouclage de production} est le premier projet réalisé pour un client externe. La \df se devait de montrer l'intérêt de son offre et ses compétences. Au délà de faire ses preuves au cabinet, c'était également une occasion de se faire une place sur le marché des entreprises de services du numérique.

Pour notre client, la \sncf, il était plutôt question de rattraper son retard sur sa transformation numérique et gagner en performance. La solution créée par ses équipes était fonctionnelle, mais témoignait d'un manque d'outils informatiques pour accomplir leurs missions.

\newpage
\subsection{Intérêt personnel}

% \begin{quote}
% Vous détaillerez l’intérêt personnel technique et/ou organisationnel acquis pendant la mission.
% \end{quote}

Cette mission m'a aussi bien apporté sur l'aspect technique que sur l'aspect organisationnel. Lors de mes missions précédentes, chez Milleis Banque, j'avais toujours travaillé sur des projets internes au sein d'un service informatique déjà bien implanté et stable. Je ressentais alors beaucoup moins le poids du planning et des délais, et je pouvais me laisser porter par la gestion de projet déjà bien rodée.

Je suis arrivé chez \tnp en étant seul avec \damien et une développeuse stagiaire qui débutait sa mission. Il y avait donc tout à construire. J'ai de ce fait été encore plus intégré aux processus de gestion de projet.
Il en va de même pour la technique, où nous avons même dû trouver une infrastructure informatique pour nos projets.

\newpage
\subsection{Conclusion et retour d'expérience sur la mission : les points perfectibles a posteriori}

% Vous indiquerez ce que vous auriez aimé améliorer : planning, choix techniques..., vous
% détaillerez aussi la pertinence de votre formation au regard de votre mission.

Avec du recul, je remarque ne m'être intéressé au planning et à la compréhension globale du projet que tardivement. Dans les premiers temps, je me suis focalisé sur les choix techniques et la conception de l'interface, de l'infrastructure et des services.

En me préparant mieux en amont -- en allant voir les plannings proposés par \damien et \gil, et l'appel d'offres par exemple --, j'aurais peut-être mieux compris le besoin et l'utilisation de l'outil par le client.

Je ne pense pas que j'aurais modifié la façon dont les plannings ont été faits cependant. Cette façon d'organiser le projet permettait de combiner la flexibilité d'un projet agile tout en se protégeant derrière des livrables clairs. En effet, les périodes de conception des lots étaient suffisamment courtes pour que le client ne fasse pas évoluer le projet pendant les phases de conception.

Ce projet, et cette année chez \tnp plus généralement, m'ont énormément appris sur le métier d'ingénieur et ont parfaitement complété ma formation à l'\epita.