La \sncf a confié à \entreprise le projet \emph{bouclage de production}.

Le « bouclage de production » permet aux directions des axes de rassembler les données des différentes applications afin de reconstituer l’ensemble de la chaîne de production ferroviaire.Après  l’ouverture  des  ventes,  cela  permet  de  s’assurer  de  la  disponibilité  des  ressources,  et d’identifier puis suivre les problèmes de l’une des briques de la chaîne de production.

Comme expliqué précédemment, le « bouclage de production » permet aux directions des axes \tgv de centraliser les données de différentes sources afin de tracer une vue synthétique de l'état de la chaîne ferroviaire. Après l'ouverture des ventes, cela permet de s'assurer de la disponibilité des ressources et également d'identifier et suivre les problèmes de disponibilité sur l'un des maillons de la chaîne.

Par exemple, si la source de données des travaux indique des travaux sur une ligne entre le 20 et le 24 mars 2020, tous les trains concernés seront mis en lumière sur l'outil de traçage.

Une solution a déjà été conçue en interne chez l'axe sud-est. Il faudrait donc s'en inspirer pour créer une solution harmonisée et plus durable pour toute la France. En effet, la solution actuelle utilise des fichiers Excel et des macros, et implique beaucoup d'actions manuelles. Ces dernières peuvent introduire des erreurs, mais la solution n'est pas à l'abri de corruption ou de perte de fichiers.

Le livrable final devrait permettre d'automatiser tous les traitements de données pour parvenir à la vue de synthèse, ainsi que d'exposer le résultat du bouclage de production sous la forme d'une interface personne-machine.
L'application devrait également permettre de réaliser des imports de données de manière autonome, s'assurer la maintenabilité du système et disposer d'une base d'archivage afin de pouvoir gérer l'historique et effectuer des analyses.