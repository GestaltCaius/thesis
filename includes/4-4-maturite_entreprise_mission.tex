\tnp est déjà familier avec le secteur des mobilités et a déjà travaillé avec la \sncf sur d'autres projets.

Cependant, il s'agit ici d'un projet de conception au sein de la \df. Cette sous-offre est récente et \tnp n'avait jamais travaillé avec un client sur cette problématique.\\
Le cabinet connaît donc bien le secteur mais très peu les problématiques de conception de projet informatique.

Cependant, l'équipe data science qui s'est occupé de toute la partie traitement de données existait avant la création de la sous-offre et possédait déjà une expérience.

Malgré la jeunesse du service, les équipes n'étaient pas pour autant inexpérimentées.\\
En effet, \damien, mon maître d'apprentissage, a 11 ans d’expérience dans le \emph{delivery} de solutions numériques pour de grandes entreprises. Il a été chef de projet informatique pendant 4 ans chez Crédit du Nord puis Publicis pour enfin être directeur du service numérique de Barclays pendant 3 ans.\\
Il est aujourd'hui directeur technique de la \df.

Quant au consultant à la tête de l'équipe \emph{data science} sur le projet, \artem, il travaille depuis 6 ans sur des projets data pour des grands comptes (optimisation du prix de vente, \textit{text mining}, plans d'analyse \textit{big data}, etc.).

Ainsi, même si le service est jeune et l'offre nouvelle, l'équipe a été encadrée par des managers expérimentés dans le domaine.\\
De plus, le cabinet de conseil a fait du secteur des mobilités un de ses fers de lance. Les différents consultants et partenaires ont été de bon conseil.
