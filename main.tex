\documentclass[a4paper, 12pt]{article}

%%% CONSIGNE PAR RAPPORT AU TITRE %%%

% Sur la couverture de votre rapport et de l’annexe, doivent figurer clairement votre nom, prénom,
% promotion, le nom de l’entreprise, le sujet de la mission, les logos de l’EPITA Apprentissage et de
% l’entreprise et la signature de votre Maître d’Apprentissage.

%%% CONSIGNE PAR RAPPORT AU TITRE %%%


% BEGIN VARIABLES
\usepackage{xspace}
\newcommand{\entreprise}{\textsc{TNP} Consultants\xspace}
\newcommand{\tnp}{\entreprise}
\newcommand{\df}{\textit{Digital Factory}\xspace}


\newcommand{\epita}{\textsc{EPITA}\xspace}
\newcommand{\sncf}{\textsc{SNCF}\xspace}

\newcommand{\tgv}{\textsc{TGV}\xspace}

\newcommand{\damien}{Damien \textsc{Pothier}\xspace}
\newcommand{\artem}{Artem \textsc{Kourlaiev}\xspace}
\newcommand{\stefan}{Stefan \textsc{Lucien}\xspace}
\newcommand{\gil}{Gil \textsc{Leveau}\xspace}
% END VARIABLES
\usepackage[french]{babel} % Format des dates, des noms des tables, etc.
\usepackage[T1]{fontenc} % Encodage recommandé avec le français

% Les marges
\usepackage[top=2.5cm, bottom=2.5cm, left=3cm, right=2cm, headheight=15pt]{geometry}
\usepackage{float} % placement des figures
% Les espaces
\usepackage{parskip}
\setlength{\parindent}{2em} % Alinea
\setlength{\parskip}{1.75em} % Interligne entre chaque paragraphes
\usepackage{setspace}

\usepackage[bottom]{footmisc} % make footnotes stick to bottom of pages

\usepackage{lmodern} % font
% \usepackage{times} % font
\usepackage{hyperref} % sommaire cliquable
\hypersetup{
    pdftitle={Mémoire d'apprentissage - Rodolphe GUILLAUME - TNP Consultants - Bouclage de production},
    pdfauthor={Rodolphe GUILLAUME},
    colorlinks=true, %colorise les liens
    breaklinks=true, %permet le retour à la ligne dans les liens trop longs
    urlcolor= blue, %couleur des hyperliens
    % linkcolor= black, %couleur des liens internes
    linkcolor= blue, %couleur des liens internes
    citecolor=blue,    %couleur des liens de citations
    filecolor=blue,      
    % bookmarks=true,
    % bookmarksopen=true,
    % pdftoolbar=false,
    % pdfmenubar=true,
    % pdfpagemode=FullScreen,
}
\usepackage{newclude} % include sub latex files

% references
\usepackage{csquotes} % needed when using biblatex with babel
\usepackage[style = chem-acs]{biblatex}
\bibliography{references}

%%% TITRE %%%
\usepackage{titling} % Permet d'ajouter un sous-titre et image EPITA
\usepackage{graphicx} % images en general (et aussi dans le titre)


% GLOSSAIRE
\usepackage[xindy, toc]{glossaries}
\makeglossaries
\loadglsentries{glossary}


% GLOSSAIRE


%%% BEGIN TITRE %%%
\newcommand{\subtitle}[1]{
    \posttitle{
        \par\end{center}
        \begin{center}
            \large#1
        \end{center}
        \vskip
        \section{0.5em}
    }
}

\pretitle{
    \begin{center}
        \LARGE
        \includegraphics[width=0.49\linewidth]{img/logo_apprentissage.png}
        \includegraphics[width=0.49\linewidth]{img/tnp_logo.jpg}\\
        [\bigskipamount]
    }
    \posttitle{
        \end{center}
    }
  
\title{
    Mémoire d'apprentissage\\
    \entreprise\\
    Bouclage de production
}

\author{
    Rodolphe \textsc{Guillaume} \\
    \epita 2020
}

\date{\today}

%%% END TITRE %%%

%%% HEADER and FOOTER %%%
\usepackage{fancyhdr}
\pagestyle{fancy}
\fancyhf{}
\rhead{Rodolphe \textsc{GUILLAUME}}
\lhead{\tnp}
\cfoot{\thepage}

\begin{document}

\onehalfspacing % Set line spacing to 1.5

\renewcommand{\listfigurename}{Table des illustrations}
\renewcommand{\figurename}{Illustration}
\renewcommand{\tablename}{Tableau}
\renewcommand{\contentsname}{Sommaire}

\maketitle
\thispagestyle{empty}
\newpage{}

% \begin{abstract}
% Ce document s'intéresse aux recommandations et bonnes pratiques de génération de clé de chiffrement en tant que particuliers, étudiants ou développeurs en entreprise. Notamment à celles du \textit{National Institute of Standards and Technology} (\textsc{NIST})
% \end{abstract}

\tableofcontents
\thispagestyle{empty}
\newpage{}

\setcounter{page}{1}

\section{Remerciements}

    \include*{includes/remerciements}
    \newpage{}

\section{Résumé de la mission}

    \include*{includes/3-resume_mission}
    \newpage{}

\section{Introduction}

% Ce chapitre doit permettre au jury de comprendre le contexte et la complexité de votre mission.

\subsection{Rappel de la mission et des finalités de celle-ci}

    \include*{includes/4-1-mission_finalites}
    \newpage{}

% Vous rappellerez l’intitulé de la mission, l’expliciterez et préciserez ses finalités.

% \subsection{4.2. Explication sur une éventuelle modification de la mission initiale}

%     \include*{includes/modifications_mission_initiale}
%     \newpage{}

% S’il y a lieu, vous expliquerez ce qui a fait évoluer le sujet. Si la variation est très importante
% vous aurez préalablement revalidé la mission avec la direction des études.
% 18/10/2019 Consignes Mémoire de fin d'apprentissage Promotion 2020 Page 8 de 17

\subsection{Présentation de l'entreprise}

    \include*{includes/4-3-presentation_entreprise}
    \newpage{}

% Vous situerez d’abord l’entreprise dans son contexte concurrentiel et dans ses perspectives, enfin vous
% situerez votre mission dans le contexte de l’entreprise et de l’organisation. Soyez imaginatif et créatif
% dans cette présentation, évitez les copier-coller des rapports d’activité, préférez votre synthèse imagée,
% originale et pertinente. Positionnez-vous dans l’entreprise.

\subsection{Maturité de l’entreprise sur les thématiques de la mission}

% Vous indiquerez les connaissances et les relations de votre mission avec le métier de l’entreprise et ses
% équipes (y compris de votre Maître d’Apprentissage)

    \include*{includes/4-4-maturite_entreprise_mission}
    \newpage{}


\subsection{État des connaissances sur la mission chez l’apprenti}

% \begin{quote}
% Vous appliquerez la même approche avec vous, vous détaillerez en quoi votre mission se rapporte à
% votre cursus EPITA par apprentissage et/ou vos expériences, et vous indiquerez vos principales
% motivations vis à vis de cette mission.
% \end{quote}

    \include*{includes/4-5-etat_connaissances_apprenti}
    \newpage{}


\subsection{Justification de l’intérêt et du positionnement de la mission pour l’entreprise.}

    \include*{includes/4-6-interet_mission_entreprise}
    \newpage{}

\subsection{Contexte précis de travail}

% \begin{quote}
% Vous développerez votre contexte de travail, en précisant les moyens fournis par l’entreprise,
% l’accessibilité des documentations, disponibilité des personnes compétentes, etc. Le but est de décrire
% ce que vous avez utilisé et comment cela a contribué efficacement à la réalisation de votre mission.
% Vous pouvez aussi mentionner les apports externes.
% \end{quote}

    \include*{includes/4-7-contexte_travail}
    \newpage{}


\section{Aspects organisationnels}

% \begin{quote}
% Le but de ce chapitre est de permettre au jury de comprendre les conditions humaines et
% organisationnelles dans lesquelles s’est déroulée votre mission, d’en mesurer la complexité et
% d’évaluer votre adaptabilité à la vie de l’entreprise.
% Pour les missions effectuées en ESN ou dans les projets, mettre en exergue ce qui relève de la mission
% et ce qui relève de l’organisation spécifique de l’ESN ou du projet.
% \end{quote}

\subsection{Découpage de la mission}

% \begin{quote}
% Vous mentionnerez les étapes et les livrables de votre mission selon les objectifs qui composent
% votre travail et si possible les conditions de passage d’une étape à la suivante. Vous essayerez
% de mentionner les étapes en série et/ou en parallèle à l’aide d’un diagramme de Gantt ou
% d’activité. Ce découpage devra être exhaustif au regard de l’ensemble des tâches accomplies
% durant la mission (fonction…).
% Mentionner également les processus/procédures qualité mis en œuvre pour la mission.
% \end{quote}

    \include*{includes/5-1-decoupage_mission}
    \newpage{}
    
\subsection{Respect des délais et critique de ce découpage}

% \begin{quote}
% Vous indiquerez comment vous avez ou non respecté le planning et le justifierez. Enfin vous
% analyserez la pertinence de ce découpage au regard des résultats obtenus et les améliorations
% qui auraient pu être envisagées.
% \end{quote}

    \include*{includes/5-2-respect-delais}
    \newpage{}
    
\subsection{Nature et fréquence des points de contrôle en interne}

% \begin{quote}
% Vous relaterez ici comment se sont enchaînées les étapes et les points de contrôle de votre
% mission ; les éventuelles revues internes avec ou sans le client, les éventuels audits, les points
% hebdomadaires, mensuels avec votre maître d’apprentissage, etc.
% \end{quote}

    \include*{includes/5-3-points-controle-interne}
    \newpage{}
    
% \subsection{Gestion des situations de crises de problèmes techniques ou budgétaire ou politiques et relationnels}

% \begin{quote}
% En cas de dépassement net des délais, vous expliquerez comment vous avez géré ces aspects
% et comment vous avez atteint/réduit vos objectifs. Vous préciserez le rôle de votre maître
% d’apprentissage et d’autres collègues dans la résolution de problèmes.
% \end{quote}

    % \include*{includes/5-4-gestion-crise}
    % \newpage{}
    
\newpage{}
\section{Aspects scientifiques, techniques et méthodologiques}

% \begin{quote}
% Ce chapitre doit permettre au jury de comprendre la complexité scientifique et/ou technique de la
% solution que vous avez du mettre en œuvre et de comprendre l’analyse qui vous a permis de bâtir cette
% solution. Il est possible d’interpréter le mot technique comme étant une démarche d’ingénieur mise
% en œuvre par l’apprenti.
% On attendra un fort contenu scientifique et donc autour de 30 pages, 35 pages maximum.

% Pour chacune de vos réalisations techniques, méthodologiques ou managériales au cours de la mission
% vous devrez évoquer les points suivants :
% \begin{itemize}
% \item Présentation du ou des objectif(s)
% \item Des alternatives possibles (à justifier)
% \item Le cadre imposé par l’entreprise
% \item Vos propositions retenues ou non
% \item Les difficultés éventuelles
% \item Le ou les résultats obtenus (s) ou le niveau d’avancement ainsi que l’impact sur le sujet et votre investissement
% \end{itemize}
% \end{quote}

    \include*{includes/6-aspects-scientifiques}
    \newpage{}
    
\newpage{}
\section{Bilan}

    \include*{includes/conclusion}
    \newpage

% \section{Bibliographie – Glossaire – Index – Table des illustrations}

% Ce chapitre doit permettre au jury de comprendre les termes utilisés dans votre mémoire et de
% connaître les sources d’information vous ayant permis de mener à bien votre démarche d’ingénieur.
% La bibliographie ne doit pas être une simple liste d’ouvrages de référence, pour chaque source on se
% doit d’expliquer en quoi elle a servi à résoudre une problématique de la mission. Pour le glossaire,
% 18/10/2019 Consignes Mémoire de fin d'apprentissage Promotion 2020 Page 11 de 17

% chaque entreprise utilise des sigles, des terminologies « métiers », etc. Ces sigles et ces termes doivent
% être expliqués et le contexte de leur utilisation précisé.

% References, glossaire

% Explications pour l'utilisation du glossaire:
% http://blog.dorian-depriester.fr/latex/utilisation-du-package-glossaries

\newpage{}

% \thispagestyle{empty}
\pagestyle{empty}
\printglossary[title={Glossaire}, toctitle={Glossaire}]
\newpage{}

% \thispagestyle{empty}
% \cleardoublepage
\phantomsection
\addcontentsline{toc}{section}{\listfigurename}
\listoffigures
\newpage{}

% \thispagestyle{empty}
% \bibliographystyle{ieeetr}
% \bibliographystyle{ieeetr}
% \bibliographystyle{unsrt}
% \cleardoublepage
\phantomsection
\addcontentsline{toc}{section}{\refname}
\printbibliography
\newpage{}

% \section{Annexes (Nombre de pages illimité, et à rendre dans un document physiquement séparé du rapport)}

% % Ce chapitre est obligatoire et doit être significatif, consistant et pertinent.
% % Le but de ce chapitre est de permettre au jury de se référer en cas de besoin aux résultats de votre
% % travail et à des présentations plus complètes d’outils, de normes. Veillez à ne pas faire un catalogue
% % inutile qui pourrait perdre le jury - pas de code source. S’assurer que chaque annexe soit bien
% % référencée dans votre rapport, et justifiez sa pertinence par quelques mots d’introduction.

% \subsection{Sommaire des annexes}

% \subsection{Documentation sur l’entreprise}

% \subsection{Documentation sur le matériel/les logiciels}

% \subsection{Etc}

\begin{appendices}

\section{Présentation de l'offre Digital \& Solutions}

\tnp possède de nombreuses offres et sous-offres, ce qui peut parfois être déroutant ou peu clair. D'ailleurs, c'est également le cas pour les consultants du cabinet eux-mêmes, qui ne connaissent pas toujours parfaitement toutes les offres de l'entreprise.

Ce document joint en annexe est justement une présentation de l'offre Digital \& Solutions -- dont la \df est une sous-offre -- à destination des consultants.

%   \includepdf[scale=0.8,pages={1},pagecommand=\section{Présentation de l'offre Digital \& Solutions}]{src/Offre digital and solutions TNP_20200131.pdf}
  \includepdf[scale=0.8,pages={1-}]{src/Offre digital and solutions TNP_20200131.pdf}
%   \includepdf[scale=0.8,pages={2-}]{src/Offre digital and solutions TNP_20200131.pdf}
 
\end{appendices}

\end{document}
