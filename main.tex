\documentclass[a4paper, 12pt]{article}
\usepackage[french]{babel} % Format des dates, des noms des tables, etc.
\usepackage[T1]{fontenc} % Encodage recommandé avec le français

% Les marges
\usepackage[top=2.5cm, bottom=2.5cm, left=3cm, right=2cm, headheight=15pt]{geometry}

% Les espaces
\usepackage{parskip}
\setlength{\parindent}{2em} % Alinea
\setlength{\parskip}{1.75em} % Interligne entre chaque paragraphes
\usepackage{setspace}

% \usepackage{lmodern} % font
\usepackage{times} % font
\usepackage{hyperref} % sommaire cliquable
\usepackage{newclude} % include sub latex files

% references
\usepackage{csquotes} % needed when using biblatex with babel
\usepackage[style = chem-acs]{biblatex}
\bibliography{references}

%%% TITRE %%%
\usepackage{titling} % Permet d'ajouter un sous-titre et image EPITA
\usepackage{graphicx} % images en general (et aussi dans le titre)

% BEGIN VARIABLES
\usepackage{xspace}
\newcommand{\entreprise}{\textsc{TNP} Consultants\xspace}
\newcommand{\epita}{\textsc{EPITA}\xspace}
\newcommand{\sncf}{\textsc{SNCF}\xspace}
\newcommand{\tgv}{\textsc{TGV}\xspace}
% END VARIABLES

% GLOSSAIRE

\usepackage[xindy, toc]{glossaries}
\makeglossaries
\loadglsentries{glossary}


% GLOSSAIRE

%%% CONSIGNE PAR RAPPORT AU TITRE %%%

% Sur la couverture de votre rapport et de l’annexe, doivent figurer clairement votre nom, prénom,
% promotion, le nom de l’entreprise, le sujet de la mission, les logos de l’EPITA Apprentissage et de
% l’entreprise et la signature de votre Maître d’Apprentissage.

%%% CONSIGNE PAR RAPPORT AU TITRE %%%

%%% BEGIN TITRE %%%
\newcommand{\subtitle}[1]{
    \posttitle{
        \par\end{center}
        \begin{center}
            \large#1
        \end{center}
        \vskip
        \section{0.5em}
    }
}

\pretitle{
    \begin{center}
        \LARGE
        \includegraphics[width=0.49\linewidth]{img/logo_apprentissage.png}
        \includegraphics[width=0.49\linewidth]{img/tnp_logo.jpg}\\
        [\bigskipamount]
    }
    \posttitle{
        \end{center}
    }
  
\title{
    Mémoire d'apprentissage\\
    \entreprise\\
    Bouclage de production
}

\author{
    Rodolphe \textsc{Guillaume} \\
    \epita 2020
}

\date{\today}
%%% END TITRE %%%

%%% HEADER and FOOTER %%%
\usepackage{fancyhdr}
\pagestyle{fancy}
\fancyhf{}
\rhead{Rodolphe \textsc{GUILLAUME}}
\lhead{\textsc{TNP}}
\cfoot{\thepage}
%%% HEADER and FOOTER %%%


\begin{document}

\onehalfspacing % Set line spacing to 1.5

\renewcommand{\listfigurename}{Table des illustrations}
\renewcommand{\figurename}{Illustration}
\renewcommand{\tablename}{Tableau}
\renewcommand{\contentsname}{Sommaire}

\maketitle
\thispagestyle{empty}
\newpage{}

\begin{abstract}
Ce document s'intéresse aux recommandations et bonnes pratiques de génération de clé de chiffrement en tant que particuliers, étudiants ou développeurs en entreprise. Notamment à celles du \textit{National Institute of Standards and Technology} (\textsc{NIST})
\end{abstract}

\tableofcontents
\thispagestyle{empty}
\newpage{}

\setcounter{page}{1}

\section{2. REMERCIEMENTS}

    \include*{includes/remerciements}
    \newpage{}

\section{3. RESUME DE LA MISSION (2 PAGES)}

    \include*{includes/3-resume_mission}
    \newpage{}

\section{4. INTRODUCTION : 10 PAGES MAXIMUM}

% Ce chapitre doit permettre au jury de comprendre le contexte et la complexité de votre mission.

\subsection{4.1. Rappel de la mission et des finalités de celle-ci (15 lignes minimum)}

    \include*{includes/4-1-mission_finalites}

Vous rappellerez l’intitulé de la mission, l’expliciterez et préciserez ses finalités.

\subsection{4.2. Explication sur une éventuelle modification de la mission initiale}

    \include*{includes/modifications_mission_initiale}
    \newpage{}

S’il y a lieu, vous expliquerez ce qui a fait évoluer le sujet. Si la variation est très importante
vous aurez préalablement revalidé la mission avec la direction des études.
18/10/2019 Consignes Mémoire de fin d'apprentissage Promotion 2020 Page 8 de 17

\subsection{4.3. Présentation de l'entreprise (3 pages maximum)}

    \include*{includes/presentation_entreprise}
    \newpage{}

Vous situerez d’abord l’entreprise dans son contexte concurrentiel et dans ses perspectives, enfin vous
situerez votre mission dans le contexte de l’entreprise et de l’organisation. Soyez imaginatif et créatif
dans cette présentation, évitez les copier-coller des rapports d’activité, préférez votre synthèse imagée,
originale et pertinente. Positionnez-vous dans l’entreprise.

\subsection{4.4. Maturité de l’entreprise sur les thématiques de la mission}

    \include*{includes/maturite_entreprise_mission}
    \newpage{}

Vous indiquerez les connaissances et les relations de votre mission avec le métier de l’entreprise et ses
équipes (y compris de votre Maître d’Apprentissage)

\subsection{4.5. Etat des connaissances sur la mission chez l’apprenti}

    \include*{includes/etat_connaissances_apprenti}
    \newpage{}

Vous appliquerez la même approche avec vous, vous détaillerez en quoi votre mission se rapporte à
votre cursus EPITA par apprentissage et/ou vos expériences, et vous indiquerez vos principales
motivations vis à vis de cette mission.

\subsection{4.6 Justification de l’intérêt et du positionnement de la mission pour l’entreprise.}

    \include*{includes/interet_mission_entreprise}
    \newpage{}

\subsection{4.7. Contexte précis de travail}

    \include*{includes/contexte_travail}
    \newpage{}

Vous développerez votre contexte de travail, en précisant les moyens fournis par l’entreprise,
l’accessibilité des documentations, disponibilité des personnes compétentes, etc. Le but est de décrire
ce que vous avez utilisé et comment cela a contribué efficacement à la réalisation de votre mission.
Vous pouvez aussi mentionner les apports externes.

\section{5. ASPECTS ORGANISATIONNELS : 25 PAGES MAXIMUM}

Le but de ce chapitre est de permettre au jury de comprendre les conditions humaines et
organisationnelles dans lesquelles s’est déroulée votre mission, d’en mesurer la complexité et
d’évaluer votre adaptabilité à la vie de l’entreprise.
Pour les missions effectuées en ESN ou dans les projets, mettre en exergue ce qui relève de la mission
et ce qui relève de l’organisation spécifique de l’ESN ou du projet.

\subsection{5.1. Découpage de la mission}

Vous mentionnerez les étapes et les livrables de votre mission selon les objectifs qui composent
votre travail et si possible les conditions de passage d’une étape à la suivante. Vous essayerez
de mentionner les étapes en série et/ou en parallèle à l’aide d’un diagramme de Gantt ou
18/10/2019 Consignes Mémoire de fin d'apprentissage Promotion 2020 Page 9 de 17
d’activité. Ce découpage devra être exhaustif au regard de l’ensemble des tâches accomplies
durant la mission (fonction…).
Mentionner également les processus/procédures qualité mis en œuvre pour la mission.

    \include*{includes/presentation_entreprise}
    \newpage{}
    
\subsection{5.2. Respect des délais et critique de ce découpage}

Vous indiquerez comment vous avez ou non respecté le planning et le justifierez. Enfin vous
analyserez la pertinence de ce découpage au regard des résultats obtenus et les améliorations
qui auraient pu être envisagées.

    \include*{includes/presentation_entreprise}
    \newpage{}
    
\subsection{5.3. Nature et fréquence des points de contrôle en interne}

Vous relaterez ici comment se sont enchaînées les étapes et les points de contrôle de votre
mission ; les éventuelles revues internes avec ou sans le client, les éventuels audits, les points
hebdomadaires, mensuels avec votre maître d’apprentissage, etc.

    \include*{includes/presentation_entreprise}
    \newpage{}
    
\subsection{5.4. Gestion des situations de crises de problèmes techniques ou budgétaire ou politiques et relationnels}

En cas de dépassement net des délais, vous expliquerez comment vous avez géré ces aspects
et comment vous avez atteint/réduit vos objectifs. Vous préciserez le rôle de votre maître
d’apprentissage et d’autres collègues dans la résolution de problèmes.

    \include*{includes/presentation_entreprise}
    \newpage{}
    
\newpage{}
\section{6. ASPECTS SCIENTIFIQUES \& TECHNIQUES ET METHODOLOGIQUES :}

Ce chapitre doit permettre au jury de comprendre la complexité scientifique et/ou technique de la
solution que vous avez du mettre en œuvre et de comprendre l’analyse qui vous a permis de bâtir cette
solution. Il est possible d’interpréter le mot technique comme étant une démarche d’ingénieur mise
en œuvre par l’apprenti.
On attendra un fort contenu scientifique et donc autour de 30 pages, 35 pages maximum.

Pour chacune de vos réalisations techniques, méthodologiques ou managériales au cours de la mission
vous devrez évoquer les points suivants :
A. Présentation du ou des objectif(s)
B. Des alternatives possibles (à justifier)
C. Le cadre imposé par l’entreprise
D. Vos propositions retenues ou non
E. Les difficultés éventuelles
18/10/2019 Consignes Mémoire de fin d'apprentissage Promotion 2020 Page 10 de 17

F. Le ou les résultats obtenus (s) ou le niveau d’avancement ainsi que l’impact sur le sujet et votre
investissement

    \include*{includes/presentation_entreprise}
    \newpage{}
    
\newpage{}
\section{7. PREMIER BILAN : 10 PAGES MAXIMUM}

Repositionnez la problématique de l’entreprise ou du sujet de la mission par rapport à un état de
l’art du marché. Il s’agira par exemple dans le cas d’un produit, d’une méthode d’un standard, de le
positionner par rapport à d’autres alternatives en démontrant votre maîtrise du domaine.
Le but de ce chapitre est de permettre au jury d’identifier et de quantifier le reste à faire d’ici la fin de
votre mission et les résultats à venir et de mesurer la valeur ajoutée que vous avez apportée à
l’entreprise et de comprendre en quoi vous avez répondu au besoin de l’entreprise.

\subsection{7.1. Intérêt de la mission pour l'entreprise/Evaluation de votre contribution à l’entreprise}

Vous détaillerez l’intérêt à court et long terme de la mission pour l’entreprise, les
perspectives ouvertes par ce sujet et la valeur ajoutée qu’elle peut en retirer pour le produit,
le service et l’entreprise - dans le cas d’une SSII, vous mettrez en évidence ces mêmes
perspectives tant pour la société de service que pour le client final.

    \include*{includes/presentation_entreprise}
    \newpage{}
    
\subsection{7.2. Intérêt personnel}

Vous détaillerez l’intérêt personnel technique et/ou organisationnel acquis pendant la mission.

    \include*{includes/presentation_entreprise}
    \newpage{}
    
\subsection{7.3. Conclusion et retour d'expérience sur la mission : les points perfectibles a posteriori}

Vous indiquerez ce que vous auriez aimé améliorer : planning, choix techniques..., vous
détaillerez aussi la pertinence de votre formation au regard de votre mission.

    \include*{includes/presentation_entreprise}
    \newpage{}
    
\newpage{}
\section{8. BIBLIOGRAPHIE – GLOSSAIRE – INDEX – TABLE DES ILLUSTRATIONS}

Ce chapitre doit permettre au jury de comprendre les termes utilisés dans votre mémoire et de
connaître les sources d’information vous ayant permis de mener à bien votre démarche d’ingénieur.
La bibliographie ne doit pas être une simple liste d’ouvrages de référence, pour chaque source on se
doit d’expliquer en quoi elle a servi à résoudre une problématique de la mission. Pour le glossaire,
18/10/2019 Consignes Mémoire de fin d'apprentissage Promotion 2020 Page 11 de 17

chaque entreprise utilise des sigles, des terminologies « métiers », etc. Ces sigles et ces termes doivent
être expliqués et le contexte de leur utilisation précisé.

% References, glossaire

% Explications pour l'utilisation du glossaire:
% http://blog.dorian-depriester.fr/latex/utilisation-du-package-glossaries

\newpage{}

\printglossary[title={Glossaire}, toctitle={Glossaire}]
\newpage{}

%\thispagestyle{empty}
\listoffigures
\newpage{}

%\thispagestyle{empty}
% \bibliographystyle{ieeetr}
% \bibliographystyle{ieeetr}
%\bibliographystyle{unsrt}
\printbibliography
\newpage{}

\section{9. ANNEXES (NOMBRE DE PAGES ILLIMITEES, ET A RENDRE DANS UN DOCUMENT PHYSIQUEMENT SEPARE DU RAPPORT)}

Ce chapitre est obligatoire et doit être significatif, consistant et pertinent.
Le but de ce chapitre est de permettre au jury de se référer en cas de besoin aux résultats de votre
travail et à des présentations plus complètes d’outils, de normes. Veillez à ne pas faire un catalogue
inutile qui pourrait perdre le jury - pas de code source. S’assurer que chaque annexe soit bien
référencée dans votre rapport, et justifiez sa pertinence par quelques mots d’introduction.

\subsection{9.1. Sommaire des Annexes}

\subsection{9.2. Documentation sur l’Entreprise}

\subsection{9.3. Documentation sur le matériel/les logiciels}

\subsection{9.4. Etc}


\end{document}
