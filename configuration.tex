\usepackage[french]{babel} % Format des dates, des noms des tables, etc.
\usepackage[T1]{fontenc} % Encodage recommandé avec le français

% Les marges
\usepackage[top=2.5cm, bottom=2.5cm, left=3cm, right=2cm, headheight=15pt]{geometry}
\usepackage{float} % placement des figures
% Les espaces
\usepackage{parskip}
\setlength{\parindent}{2em} % Alinea
\setlength{\parskip}{1.75em} % Interligne entre chaque paragraphes
\usepackage{setspace}

\usepackage[bottom]{footmisc} % make footnotes stick to bottom of pages

\usepackage{lmodern} % font
% \usepackage{times} % font
\usepackage{hyperref} % sommaire cliquable
\hypersetup{
    pdftitle={Mémoire d'apprentissage - Rodolphe GUILLAUME - TNP Consultants - Bouclage de production},
    pdfauthor={Rodolphe GUILLAUME},
    colorlinks=true, %colorise les liens
    breaklinks=true, %permet le retour à la ligne dans les liens trop longs
    urlcolor= blue, %couleur des hyperliens
    % linkcolor= black, %couleur des liens internes
    linkcolor= blue, %couleur des liens internes
    citecolor=blue,    %couleur des liens de citations
    filecolor=blue,      
    % bookmarks=true,
    % bookmarksopen=true,
    % pdftoolbar=false,
    % pdfmenubar=true,
    % pdfpagemode=FullScreen,
}
\usepackage{newclude} % include sub latex files

% references
\usepackage{csquotes} % needed when using biblatex with babel
\usepackage[style = chem-acs]{biblatex}
\bibliography{references}

%%% TITRE %%%
\usepackage{titling} % Permet d'ajouter un sous-titre et image EPITA
\usepackage{graphicx} % images en general (et aussi dans le titre)


% GLOSSAIRE
\usepackage[xindy, toc]{glossaries}
\makeglossaries
\loadglsentries{glossary}


% GLOSSAIRE


%%% BEGIN TITRE %%%
\newcommand{\subtitle}[1]{
    \posttitle{
        \par\end{center}
        \begin{center}
            \large#1
        \end{center}
        \vskip
        \section{0.5em}
    }
}

\pretitle{
    \begin{center}
        \LARGE
        \includegraphics[width=0.49\linewidth]{img/logo_apprentissage.png}
        \includegraphics[width=0.49\linewidth]{img/tnp_logo.jpg}\\
        [\bigskipamount]
    }
    \posttitle{
        \end{center}
    }
  
\title{
    Mémoire d'apprentissage\\
    \entreprise\\
    Bouclage de production
}

\author{
    Rodolphe \textsc{Guillaume} \\
    \epita 2020
}

\date{\today}

%%% END TITRE %%%

%%% HEADER and FOOTER %%%
\usepackage{fancyhdr}
\pagestyle{fancy}
\fancyhf{}
\rhead{Rodolphe \textsc{GUILLAUME}}
\lhead{\tnp}
\cfoot{\thepage}